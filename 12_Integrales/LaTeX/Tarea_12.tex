\documentclass[10pt,letterpaper]{article}
\usepackage[utf8x]{inputenc}
\usepackage{ucs}
\usepackage[spanish]{babel}
\usepackage{amsmath}
\usepackage{amsfonts}
\usepackage{amssymb}
\usepackage{graphicx}
\usepackage[left=0.5cm,right=0.5cm,top=0.5cm,bottom=0.5cm]{geometry}
\author{F\'elix Ernesto Charry Pastrana}
\sloppy
\setlength{\parindent}{0pt}
\usepackage[none]{hyphenat}
\begin{document}
\textbf{Universidad Nacional Autónoma de México} \\
Presentado a: \textbf{Santiago Caballero} \\
Presentado por: \textbf{Félix Ernesto Charry Pastrana }\\
Maestría en Ciencias Físicas \\
Introduccion a la fisica computacional \\
2018.03.15 \\
\\
{\Large \textbf{Tarea 12}}
\\
\\
Resultados: 
\\
1) 
\[
f(x) = 2\sin(x) + \cos(2x)
\]
\[
\int_0^5 f(x) \,dx \simeq 1.160665 
\]
2) 
\[
g(x) = \dfrac{\sin(\pi\,x)}{\pi\,x}
\]
\[
2\,\int_0^10 g(x) \,dx \simeq 0.979757
\]
3) 
\[
h(x) = J_0(x), \hspace{1cm} z_{0,3}\text{ : tercer zero de la funcion}
\]
\[
\int_0^{z_{0,3}} h(x) \,dx \simeq 1.268168
\]

\end{document}