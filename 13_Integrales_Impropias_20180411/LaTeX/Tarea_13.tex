\documentclass[10pt,letterpaper]{article}
\usepackage[utf8x]{inputenc}
\usepackage{ucs}
\usepackage[spanish]{babel}
\usepackage{amsmath}
\usepackage{amsfonts}
\usepackage{amssymb}
\usepackage{graphicx}
\usepackage[left=0.5cm,right=0.5cm,top=0.5cm,bottom=0.5cm]{geometry}
\author{F\'elix Ernesto Charry Pastrana}
\sloppy
\setlength{\parindent}{0pt}
\usepackage[none]{hyphenat}
\begin{document}
\textbf{Universidad Nacional Autónoma de México} \\
Presentado a: \textbf{Santiago Caballero} \\
Presentado por: \textbf{Félix Ernesto Charry Pastrana }\\
Maestría en Ciencias Físicas \\
Introduccion a la fisica computacional \\
2018.04.12 \\
\\
{\Large \textbf{Tarea 13}}
\\
\\
Resultados: 
\\
1) 
\[
f_1(x) = \ln(x)\,\ln(1-x)
\]
\[
\int_{0}^{1} f_1(x) \,dx \simeq 0.355065 
\]
2) 
\[
f_2(x) = \dfrac{1}{\sqrt{x} (1+x)}
\]
\[
\int_{0}^{\infty} f_2(x) \,dx \simeq 3.141592 
\]
3) 
\[
f_3(x) = x^{-\frac{3}{2}}\,\sin\left(\dfrac{x}{2}\right)\,\exp(-x)
\]
\[
\int_{0}^{\infty} f_3(x) \,dx \simeq 0.861179 
\]
4) 
\[
f_4(x) = x^{-\frac{2}{7}}\,\exp(-x^2)
\]
\[
\int_{0}^{\infty} f_4(x) \,dx \simeq 1.246631 
\]

\end{document}